\documentclass[a4paper]{article}
\usepackage[margin=1in]{geometry} % 设置边距,符合Word设定
\usepackage{ctex}
\usepackage{graphicx} %插入图片的宏包
\usepackage{float} %设置图片浮动位置的宏包
\usepackage{subfigure} %插入多图时用子图显示的宏包
\usepackage{lipsum}
\usepackage{minted}% 语法高亮和代码样式设置方面更加强大和灵活
\usepackage{listings}% 引入listings包,用于在文档中插入代码,并可自定义代码样式
\usepackage{caption}



\title{奥运会项目评估模型:随机森林回归在体育决策中的应用}
\author{ Guo Runheng,\ Yang Zicheng,\ Qiu Xinyang}
\date{\today}

\begin{document}
    \maketitle


\begin{abstract}
    % 随着奥运会项目的不断演变,国际奥林匹克委员会(IOC)面临着选择哪些运动项目(SDEs)加入2032年夏季奥运会的挑战。本研究旨在通过构建一个数学模型,根据IOC的标准评估SDEs,以提供有依据的推荐。该模型综合考虑了受欢迎度和可达性、性别平等、可持续性、包容性、相关性与创新以及安全与公平竞争等多个维度。通过对历史数据的分析和模型的构建,我们识别出可能在2032年布里斯班奥运会上新增或重新引入的SDEs,并对其优先级进行排序。本研究的结果将为IOC提供决策支持,确保奥运会项目的相关性和影响力。
    随着奥运会项目的不断演变,国际奥林匹克委员会(IOC)面临着选择哪些运动项目(SDEs)加入2032年夏季奥运会的挑战。
    \par 针对任务1,我们根据IOC标准,通过在网上查找多方面的资料并参考权威数据,对这些运动的受欢迎度和可达性、性别平等性、可持续性、包容性、相关性与创新性、安全与公平竞争性、新建更多项目的可能性共七个维度因素,在0-1之间进行了评分。
	\par 针对任务2,我们对1948-2020年项目数量的变化表格进行分析,在排除了一些受极端事件影响、较多年份数据残缺的项目后,选出33个项目的数据,利用Python语言进行了数据的可视化操作。然后,我们使用一次线性拟合,分析从1948到2020年的项目数变化的规律性特征。以七个维度作为自变量,线性拟合的斜率作为因变量,训练了一个随机森林回归模型。
    \par 针对任务4,我们利用训练好的随机森林回归模型,识别了三个可能在2032年布里斯班奥运会上新增或重新引入的 SDEs ,并确定其优先推荐的顺序。
\end{abstract}
\tableofcontents
\section{Introduction}

\subsection{Background}     
\subsection{Our Work}
\subsection{Data Pre-processing}


\section{Assumptions}
我们通过在网上查找多方面的资料并参考权威数据,最后对33个项目的七个维度进行了评分,结果如下:

展示表格

\section{Abbreviation and Definitions}
\section{The Evaluate Model}
\subsection{线性拟合模型}
We first excluded some items with poor data, such as 1, 2, and 3.Finally, we chose data from 33 items to train our model, 20\% of them are randomly decided as test dataset for model, and two of them are for manually test.
We then used one linear fit to the number of projects from 1948 to 2020 for each project:
\captionsetup[listing]{labelformat=empty}
\begin{figure}[H] %H为当前位置,!htb为忽略美学标准,htbp为浮动图形
    \centering %图片居中
    \includegraphics[width=0.7\textwidth]{TrendChartOfProject} %插入图片,[]中设置图片大小,{}中是图片文件名
    \caption{Main name 2} %最终文档中希望显示的图片标题
    \label{Fig.main2} %用于文内引用的标签
    \end{figure}

\subsection{随机森林回归}
\subsubsection{Introduction}
%TODO: 随机森林回归的定义

我们以七个维度和作为自变量,线性拟合的斜率作为因变量,训练了一个随机森林回归模型。随机森林模型是一种基于决策树的模型,其本身是多个决策树并行的训练方式,有效地降低了训练的时间地同时带来更加精确的数据模型。该模型具有随机性强、不易overfit、对异常点outlier不敏感、适用于数据集中存在大量未知特征的情况、处理高维数据相对较快、能够估计哪个特征在分类中更重要、可以并行处理的优点,不过由于是黑盒,其不可解释的地方较多,在噪声过大的地方容易过拟合。
\subsubsection{Principle}
随机森林模型是基于决策树的,此处首先解释决策树选择最优决策的方法。此处引入熵的概念。\\
决策树是一种二叉树,每一层只做出一种选择,我们可以通过熵来衡量每一个父节点的不确定性。熵的计算公式如下:\\
$$H(X)= -\sum_{k=1}^N p_k log_2(p_k)$$
除去整一棵决策树的祖先,和最小的子节点,我们能算出每一个节点的熵的增益:
$$G(D)= H(X)-\sum_{v=1}^V \frac {\left|D^v\right|}{\left|D\right|} H(D)^v$$
其中G(D)表示熵的增益,H(X)表示上一层的熵,后面是当前一层的熵的总和。\\
决策树的整体流程如下:\\
(1) 过滤所有可能的决策条件。\\
(2) 选择子节点熵最小的决策条件。\\
(3) Repeat step1 and step2 until it reach the maximumpredefined depth or all the elements in a single lead node belong to one class.\\[1em]
接着来讲随机森林模型。随机森林模型首先从大量的数据中取样,然后构建多棵并行的决策树,基于决策树的熵最小决策原理得到每一棵树的最终决策,通过回归算法我们能用这些决策获得一个最终评测值,通过指定指标即可获得预测结果。
\subsubsection{Implementation}
We used the random forest fitting algorithm from the sklearn module of python.
% Code
\begin{listing}[htb]\caption{STH}\label{code:processdweet}
    \begin{minted}{python3}
        model = RandomForestRegressor(n_estimators=100, random_state=42)
        model.fit(X_train, y_train)
\end{minted} 
\end{listing}


\subsubsection{Result Analysis}
In the test set, our model achieved an R Square error of 0.6 and a mean square error of 0.12, making the regression training basically successful.

We used this model to predict data outside of the training and test sets

\begin{table}[]
    \begin{tabular}{lllllllll}
    Name      & Popularity and Accessibility & Gender Equity & Sustainability & Inclusivity & Relevance and Innovation & Safety and Fair Play & Basis & Rank        \\
    Wrestling & 0.7                          & 0.6           & 0.8            & 0.8         & 0.5                      & 0.8                  & 0     & 0.105263158 \\
              &                              &               &                &             &                          &                      &       &             \\
              &                              &               &                &             &                          &                      &       &            
    \end{tabular}
    \end{table}

结果表明,我们的模型在预测项目数量的增长上取得了成功。



\section{对SDE项目的预测}
我们对某个SDE项目进行了评分,输入我们的模型,得到了如下结果

我们认为 它的增长是xxx
\subsubsection{Background of This Problem}
\subsection{Efficiency and Robustness}
\section{Sensitivity Analysis}
\section{Strengths and Weaknesses}
\subsection{Strengths}
\subsection{Weaknesses}
\section{Conclusion}
Hello world!test123
\end{document}  