%%
%% EasyMCM: 一个简洁、易用的美赛模板
%% Oringinal template (MCMthesis v5.0) by latexstudio, <latexstudio@hotmail.com>
%% Redeveloped by youjiarui189 (xjtu-blacksmith), <yjr134@163.com>
%% 目前由西安交大钱院学辅(@qyxf)负责,网址:https://qyxf.site/
%%
%% v5.00: 注释全部汉化(且添加了大量注释),并将格式改进为符合当前规范的版本
%% v5.01: 正式改为独立宏包 EasyMCM,改为简洁、清晰、符合美赛规范的格式
%% v5.02: 支持了 7 位控制号,改进了若干已知问题,并补充了一些非必需的内容
%% v5.1 : 适配 2020 年 Summary sheet 样式,大幅优化代码,更加简明易用
%% 
%% 博客地址:http://www.cnblogs.com/xjtu-blacksmith/
%% GitHub 发布页面:https://github.com/qyxf/easymcm/releases
%%

\NeedsTeXFormat{LaTeX2e}[2005/12/01]
\ProvidesPackage{easymcm}[2019/01/21 EasyMCM v5.1]
\typeout{EasyMCM Package, version 5.1}

% 基本宏包 
\RequirePackage[a4paper,margin=2.5cm,bottom=1.5cm]{geometry}  % 页边距和纸张大小
\RequirePackage{fancyhdr,fancybox}  % 设置页眉页脚
\RequirePackage{ifthen}  % 逻辑结构
\RequirePackage{lastpage}  % 生成"Page X of XX"
\RequirePackage{paralist}  % 修改 itemize 各项间距
\RequirePackage{indentfirst}  % 全文首行缩进
\RequirePackage[toc,page,title,titletoc,header]{appendix}  % 附录定制

% 数学公式
\RequirePackage{amsfonts,amsmath,amssymb} % AMS-LaTeX 符号、公式

% 定理环境声明
% 事实上大部分论文中用不到,但仍保留以供使用
\newtheorem{Theorem}{Theorem}[section]
\newtheorem{Lemma}[Theorem]{Lemma}
\newtheorem{Corollary}[Theorem]{Corollary}
\newtheorem{Proposition}[Theorem]{Proposition}
\newtheorem{Definition}[Theorem]{Definition}
\newtheorem{Example}[Theorem]{Example}

% 目录
\RequirePackage{titlesec,titletoc}  % 自定义目录样式
\setcounter{tocdepth}{3}  % 目录显示到 subsubsection(3) 级别
\RequirePackage[font=small]{caption}  % 使图表标题字号小一号
\renewcommand\abstractname{Summary}  % 摘要标题

% 表格相关
\RequirePackage{longtable,multirow,array}  % 各种基本的表格宏包
\RequirePackage{booktabs}  % 三线表宏包
\RequirePackage{tabularx}

% 字体相关
\RequirePackage[T1]{fontenc}  % 开启拓展区正文字体
\RequirePackage{url}  % 网址宏包,以下命令使得网址能自动换行
\def\UrlBreaks{\do\A\do\B\do\C\do\D\do\E\do\F\do\G\do\H\do\I\do\J
\do\K\do\L\do\M\do\N\do\O\do\P\do\Q\do\R\do\S\do\T\do\U\do\V
\do\W\do\X\do\Y\do\Z\do\[\do\\\do\]\do\^\do\_\do\`\do\a\do\b
\do\c\do\d\do\e\do\f\do\g\do\h\do\i\do\j\do\k\do\l\do\m\do\n
\do\o\do\p\do\q\do\r\do\s\do\t\do\u\do\v\do\w\do\x\do\y\do\z
\do\.\do\@\do\\\do\/\do\!\do\_\do\|\do\;\do\>\do\]\do\)\do\,
\do\?\do\'\do+\do\=\do\#}

% 基本参数
\setlength{\headheight}{15pt}
\newcommand{\MCM@control}{0000000}  % 队伍控制号,默认为 0000000
\DeclareOption*{\edef\MCM@control{\CurrentOption}}  % 宏包选项接收控制号
\ProcessOptions
\newcommand{\control}{\MCM@control}
\newcommand{\team}{Team \#\ \MCM@control}
\newcommand{\contest}{MCM/ICM}

% 书签,插图和交叉引用的设置
\RequirePackage{graphicx}  % 插图
\RequirePackage{flafter}  % 引用该宏包可避免图片在引用它的正文之前出现
\RequirePackage{float} 
\RequirePackage{subfigure}
\RequirePackage{ifpdf}  % 判断是否在运行 pdftex
\ifpdf%
\RequirePackage{epstopdf}  % pdftex 不能使用 eps 图片,用该宏包转成 pdf 后使用
\DeclareGraphicsExtensions{.pdf,.jpg,.jpeg,.png}    % 允许的图片类型
\RequirePackage[
    linkcolor=black,  % 消除链接色彩
    citecolor=black,
    colorlinks=true,
    linkcolor=black,
    citecolor=black,
    urlcolor=black]{hyperref}
\else\DeclareGraphicsExtensions{.eps,.ps}  % 不需要转换 eps 图片格式
\ifxetex\RequirePackage[
    xetex,  % 运行 xetex
    pdfstartview=FitH,
    bookmarksnumbered=true,
    bookmarksopen=true,
    colorlinks=true,
    linkcolor=black,
    citecolor=black,
    urlcolor=black]{hyperref}
\else\RequirePackage[
    dvipdfm,  % 运行其他编译引擎(如 luatex)
    pdfstartview=FitH,
    bookmarksnumbered=true,
    bookmarksopen=true,
    colorlinks=true,
    linkcolor=black,
    citecolor=black,
    urlcolor=black]{hyperref}
\fi\fi

% 以下设置使得一页上最多有六个浮动对象(图、表)
% 且顶部最多三个,底部最多三个
% 可自行修改参数
\setcounter{totalnumber}{6}
\setcounter{topnumber}{3}
\setcounter{bottomnumber}{3}

% 很多人发现缺省的浮动参数过于严格了
% 下面的命令
\renewcommand{\textfraction}{0.15}
\renewcommand{\topfraction}{0.85}
\renewcommand{\bottomfraction}{0.65}
\renewcommand{\floatpagefraction}{0.60}
% 将浮动参数重新设置为更宽松的值。
% ---选自《LaTeX2e插图指南》

% 图表标题名称
\renewcommand{\figurename}{Figure}
\renewcommand{\tablename}{Table}
\setlength{\belowcaptionskip}{4pt}
\setlength{\abovecaptionskip}{4pt}  % 设置 caption 与上下文间距

% 以下定义了自动识别的图表文件夹
% 若使用这些名字命名文件夹
% 则引用图片路径时只需填文件名即可
\graphicspath{{./}{./img/}{./fig/}{./image/}{./figure/}{./picture/}}

% 页眉页脚设置
\lhead{\small \team}
\chead{}
\rhead{\small Page \thepage\ of \pageref{LastPage}}
\lfoot{}
\cfoot{}
\rfoot{}

% 信件/备忘录环境
\newcounter{prefix}  % 创建隐藏前缀计数器,避免对 letter 环境编号
\renewcommand{\theHsection}{\theprefix.\thesection}  % 针对 hyperref
\newenvironment{letter}[1]{\refstepcounter{section}\addtocounter{section}{-1}\section*{#1}\addcontentsline{toc}{section}{#1}}{\stepcounter{prefix}}

% 快乐 etoolbox
\RequirePackage{etoolbox}  % 减轻正文复杂度
\AtBeginEnvironment{abstract}{\setlength\parskip{1ex}}  % 摘要中增加段距
\AtBeginEnvironment{thebibliography}{
    \refstepcounter{section}
    \addcontentsline{toc}{section}{References}}  % 参考文献附加链接
\BeforeBeginEnvironment{subappendices}{
    \clearpage
    \setcounter{secnumdepth}{-1}}  % 附录附加链接
\BeforeBeginEnvironment{letter}{\clearpage}  % 信件环境附加换页

% 代码相关
\RequirePackage{listings}
\RequirePackage{color,xcolor}
\definecolor{grey}{rgb}{0.8,0.8,0.8}
\definecolor{darkgreen}{rgb}{0,0.3,0}
\definecolor{darkblue}{rgb}{0,0,0.3}
\def\lstbasicfont{\fontfamily{pcr}\selectfont\footnotesize}
\lstset{%
   % numbers=left,
   % numberstyle=\small,%
    showstringspaces=false,
    showspaces=false,%
    tabsize=4,%
    frame=lines,%
    basicstyle={\footnotesize\lstbasicfont},%
    keywordstyle=\color{darkblue}\bfseries,%
    identifierstyle=,%
    commentstyle=\color{darkgreen},%\itshape,%
    stringstyle=\color{black}%
}
\lstloadlanguages{C,C++,Java,Matlab,Mathematica}


% COMAP 要求的 Summary Sheet 标题(2020)
% 注意每年比赛时有可能有小调整
% 请以官网发布的样式为准自行做小的修改!
\newcommand{\@problem}[1]{}
\newcommand{\problem}[1]{\gdef\@problem{#1}}
\newcommand{\makesheet}{ %生成sheet头命令的定义
    \null%
    \vspace*{-5pc}%
    \begin{center}
    \begingroup
    \setlength{\parindent}{0pt}
    \begin{minipage}[t]{0.33\linewidth}
        \centering
        \textbf{Problem Chosen}\\
        \LARGE\@problem
        \end{minipage}%
        \begin{minipage}[t]{0.34\linewidth}
        \centering
        \bfseries\the\year\\\contest\\{Summary Sheet}
        \end{minipage}%
        \begin{minipage}[t]{0.33\linewidth}
        \centering
        \textbf{Team Control Number}\\
        \LARGE\MCM@control\\[1.8pc]
    \end{minipage}\par
    \vskip1ex
    \rule{\linewidth}{1.5pt}\par
    \endgroup
    \vskip 10pt%
    \end{center}}

% abstract 环境的设置
\newbox\@abstract   % 将摘要创建为盒子
\setbox\@abstract\hbox{}  % 盒子置空
\long\def\abstract{\bgroup\global\setbox\@abstract\vbox\bgroup\hsize\textwidth}
\def\endabstract{\egroup\egroup}
\def\make@abstract{
    \vskip -10pt\par
    {\centering\Large\bfseries\@title\vskip1ex}\par  % 插入论文标题,字号可自己修改
    {\centering\bfseries\abstractname\vskip1.5ex}\par  %摘要标题
    \noindent\usebox\@abstract\par  % 摘要正文
    \vskip 10pt}  % 底部留空,若不需要可删去

% Summary Sheet 生成
\def\@maketitle{
	\makesheet%
	\make@abstract
    \pagenumbering{gobble}
    \pagestyle{empty}
    \newpage
    \pagenumbering{arabic}
    \setcounter{page}{2}}%1

% 目录生成
\renewcommand\tableofcontents{%
    \centerline{\normalfont\Large\bfseries\contentsname%
    \@mkboth{%
    \MakeUppercase\contentsname}{\MakeUppercase\contentsname}}%
    \vskip 3ex%
    \@starttoc{toc}%
    \thispagestyle{fancy}
    \clearpage
    \pagestyle{fancy}
    \setlength\parskip{1ex}}  % 调整段间距

\endinput