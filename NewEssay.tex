 % COMAP2024SCUT template
 % Created by Caibin Zeng, email: macbzeng@scut.edu.cn  %
 % January 8, 2025
 %%
 
 %美赛模板:正文部分

 \documentclass[12pt]{article}  % 官方要求字号不小于 12 号,此处选择 12 号字体
 % \linespread{1.1}
 % \bibliographystyle{plain}
 % 本模板不需要填写年份,以当前电脑时间自动生成
 %%%%%%%%%%%%%%%%%%%%%%%%%%%%%%%%%%%%%%%%%%%%%%%%%%%%%%%%%%%%%%%%%%%%%%%%%%%%%%%
 % 请在以下的方括号中填写队伍控制号
 \usepackage[2511654]{easymcm}  % 载入 EasyMCM 模板文件
 \problem{B}  % 请在此处填写题号
 %%%%%%%%%%%%%%%%%%%%%%%%%%%%%%%%%%%%%%%%%%%%%%%%%%%%%%%%%%%%%%%%%%%%%%%%%%%%%%%%
 % \usepackage{mathptmx}  % 这是 Times 字体,中规中矩 
 \usepackage{palatino}  % mathpazo 这palatino是 COMAP 官方杂志采用的更好看的 Palatino 字体,可替代以上的 mathptmx 宏包
 \usepackage{ctex}
 \usepackage{pdfpages}
 \usepackage{longtable}
 \usepackage{tabu}
 \usepackage{threeparttable}
 \usepackage{listings}
 \usepackage{paralist}
 \usepackage{hyperref}
 \usepackage[linesnumbered,ruled,vlined]{algorithm2e}
 \usepackage{subfigure}
 \usepackage{subcaption}
 \usepackage{cleveref} %可以调用
 \usepackage{tcolorbox}
 
 \tcbuselibrary{most}
 \newtcolorbox{mybox}[2][]{colbacktitle=red!10!white, colback=blue!10!white,coltitle=red!70!black, title={#2},fonttitle=\bfseries,#1}
 \graphicspath{{img/}}          % 此处{img/}为相对路径,注意加上“/”
  \let\itemize\compactitem
  \let\enditemize\endcompactitem
 \newcommand{\upcite}[1]{\textsuperscript{\textsuperscript{\cite{#1}}}}
 
 
 \title{The Title Should be Concise and Informative}  % 标题
 
 % 如需要修改题头(默认为 MCM/ICM),请使用以下命令(此处修改为 MCM)
 %\renewcommand{\contest}{MCM}
 
  %文档开始
 \begin{document}
 
 % 此处填写摘要内容
 \begin{abstract}
Juneau City in Alaska, USA is facing the problem of overtourism. The number of tourists and tax rates need to be adjusted to improve the satisfaction of tourists and residents, and finally realize sustainable development of tourism industry. The government has implemented some measures to limit the number of tourists, protect the environment, and improve local infrastructure, but the results have not been satisfactory due to the complex factors affecting tourist and resident satisfaction.
     
This article establishes a new satisfaction evaluation model that comprehensively considers the impact of tourist numbers on residents' income, government spending, ecological environment, and social pressure, in order to adjust the daily number of tourists and achieve multi-objective optimization of tourist and resident satisfaction.

\textit{The main steps are as follows:}
 
First, a causal model was established to analyze the principle of how the number of tourists affects tourist satisfaction and resident satisfaction.

 Then, based on the causal model and combined with the searched data, several equations were established to construct a multi-objective optimization model for the impact of tourist numbers on tourist satisfaction and resident satisfaction.

 To maximize stakeholder satisfaction, the NSGA-II genetic algorithm is used to solve the Pareto solution of the multi-objective problem and determine the optimal number of tourists.
 
 This study successfully simulated and optimized the number of tourists in Juneau city by constructing a causal diagram model, which benefits both tourists and residents, and more importantly, can achieve sustainable development of tourism in the Juneau area. In addition, through adjustments, the model can be applied to other destinations affected by excessive tourism and has good universality.
     
 % 此处填写关键字,以分号分开
     \vspace{5pt}  %mm	毫米	1 mm = 2.845 pt   pt 点	1 pt = 0.351 mm
     \textbf{Keywords}: Causal Model; Overtourism; Multi-objective optimization; NSGA-II; Sustainable tourism;
 
 \end{abstract}
 
 \maketitle  % 生成 Summary Sheet
 
 \tableofcontents  % 生成目录
 
 
 % 正文开始
 % Chapter 1: Introduction
 \section{Introduction}
 \subsection{Background}
In recent years, the rapid development of the global tourism industry has brought many challenges, among which the problem of overtourism is particularly prominent. Take Juneau, Alaska, USA as an example. The city received about 1.6 million cruise passengers in 2023, setting a new record. On the busiest days of the tourism season, Juneau could receive up to seven large cruise ships, with as many as 20,000 visitors. Although these tourists brought in a substantial revenue of about 375 million dollars for the city, they also caused many problems, such as increased pressure on local infrastructure, tight drinking water supply, difficult waste management, and increased carbon footprint. In addition, overtourism has also had a negative impact on local culture and natural landscapes, of which the recession of Mendenhall Glacier is a typical case. Since 2007, the glacier has receded the equivalent of eight football fields, which not only caused local residents to worry about the disappearance of the glacier, but also raised questions about the sustainability of tourism. Therefore, how to maintain the economic benefits of tourism while reducing its negative impact on the environment and society, and achieve sustainable development, has become an urgent problem for Juneau to solve.
 
 \subsection{Restatement of the Problem}
 Given the context of overtourism in Juneau, Alaska, and its implications on both the environment and local community, the following issues need to be addressed:
 \begin{itemize}
    \item[$\bullet$]Develop a sustainable tourism model for Juneau, considering factors such as visitor numbers, overall revenue, and tourism stabilization measures. Clearly identify the factors to be optimized and those serving as constraints. Include a financial plan for additional revenue and demonstrate its feedback loop into the model for sustainable tourism promotion. Conduct a sensitivity analysis to highlight the most critical factors.
    \item[$\bullet$]Adapt the model to other overtourism-affected destinations, considering how location choice impacts the importance of different measures. Explore ways to use the model to promote less-visited attractions or locations to achieve a balanced tourism distribution.
    \item[$\bullet$]Draft a one-page memo to the Juneau Tourist Council, outlining predictions, the effects of various measures, and recommendations for optimizing outcomes.
 \end{itemize}
\subsection{Our Work}
 
To demonstrate our workflow visually, the flow chart is shown. 
\ref{fig1}.
  

 \begin{figure}[h]  %h此处,t页顶,b页底,p独立一页,浮动体出现的位置 [H]
 
 \centering  %图表居中
 \includegraphics[width=.9\textwidth]{chart2.png} %图片的名称或者路径之中有空格会出问题 
 \caption{Flow Chart of Our Work} % 图片标题                 
 \label{fig1}%交互引用
 \end{figure}
 
 
 \section{Assumptions and Explanations}
 
Since there're several unacquirable data, we simplify our model by making assumptions.
 
 \begin{itemize}
     \setlength{\parsep}{0ex} %段落间距
     \setlength{\topsep}{2ex} %列表到上下文的垂直距离
     \setlength{\itemsep}{1ex} %条目间距
     \item[\bfseries \textit{Assumption} 1:] All the tourists who went to Juneau had been on a cruise.
     %\item[\bfseries \textit{Explanation:}]  
     \vspace{1ex}
     \item[\bfseries \textit{Assumption} 2:] After deducting government expenses from taxes, the remaining funds are all used for infrastructure maintenance and environmental protection.
     %\item[\bfseries \textit{Explanation:}]  XXX
     \vspace{1ex}
     \item[\bfseries \textit{Assumption} 3:] Incorporate all other taxes into the consumption tax rate calculation uniformly.
     %\item[\bfseries \textit{Explanation:}]  XXX
     \vspace{1ex}
     \item[\bfseries \textit{Assumption} 4:] Juneau's revenue comes entirely from the tourism industry.
 \end{itemize}
 
 Additional assumptions are made to simplify the analysis for individual sections. These assumptions will be discussed at the appropriate locations.
 
 \section{Notations}
 Some important mathematical notations used in this paper are listed in Table \ref{tab1}. 
 \begin{table}[htbp]
 \begin{center}
 \caption{Notations used in this paper}
 \begin{tabular}{cc} % 第一列居中对齐、第二列居左对齐
 \toprule[2pt]
 \multicolumn{1}{m{4cm}}{\centering Symbol}
 &\multicolumn{1}{m{10cm}}{\centering Description }\\  %3cm 和 8cm 是列宽,根据实际需求修改
 \midrule
 $x$   &         Tax rate \\
 $T$   &         Tax \\
 $C$   &         Tourists count per year\\
 $E$   &         Consumption per tourist \\
 $TTI$ &         Tax used to maintain infrastructures\\
 $CDI$ &         Each tourist s damagement to infrastructures\\
 $IS$  &         State of infrastructures\\
 $CTS$ &         Tourist\textquotesingle s satisfaction affected by tourist count\\
 $CIS$ &         Tourist\textquotesingle s satisfaction influenced by infrastructures\\
 $LinS$ &        The Locals\textquotesingle satisfaction affected by income\\
 $LIS$ &         The Locals\textquotesingle satisfaction affected by infrastructures\\
 $LS$ &          Satisfaction of the locals\\
 $CS$ &          Satisfaction of the tourists\\
 $EIE$ &         Expenditure in maintaining infrastructures and environment\\
 \bottomrule[2pt]
 \end{tabular}	\label{tab1} % 交互引用 
  \iffalse\begin{tablenotes}
         \footnotesize
         \item[*] *\href{https://www.caam.rice.edu/~heinken/latex/symbols.pdf}{ \LaTeX~Mathematical symbols collected by Prof. M. Heinkenschloss}. %此处加入注释*信息
       \end{tablenotes}
       \fi
 \end{center}
 \end{table} 
 \vspace{-1cm} 
 
 
 
 \section{Model about Tourism and Destination}
 \subsection{关系模型的构建}
 In Juneau, the conflict between environment and the tourists can be simplified into the following model. In order to better quantify the impact of various factors on the locals and tourists, we set tourist satisfaction and local satisfaction as measuring standards.
 
 \begin{figure}[htbp]  %h此处,t页顶,b页底,p独立一页,浮动体出现的位置 [H]
 
    \centering  %图表居中
    \includegraphics[width=1.2\textwidth]{chart1.png} %图片的名称或者路径之中有空格会出问题 
    \caption{Causal Model of Tourism and Attractions} % 图片标题                 
    \label{fig1}%交互引用
    \end{figure}
 

 \begin{itemize}
     \setlength{\parsep}{0ex} %段落间距
     \setlength{\topsep}{2ex} %列表到上下文的垂直距离
     \setlength{\itemsep}{1ex} %条目间距
 \item [\textbf{Explanations:}] 
        \item Tourist count: The number of tourists. When the number up to a limited point, it will reduce the satisfaction of tourists since its overcrowded experience.
        \item Infrastructures and Environment: The state of infrastructures and the living condition. Overtourism may lead to a considerable deterioration on infrastructures in Juneau and damage on the environment.
        \item Expenditure: The expenditure of tourists, which can be divided into incomes of the locals and taxes. Part of the tax can be used to maintain infrastructures.
        \item AHR: Anthropogenic heat release.
        \item "+": Positive correlation.
        \item "-": Negative correlation.\\
 \end{itemize}
 \subsection{模型分析}
 该模型的建立参照了朱诺市相关的文献,客观分析了不同因素带来的影响。其中,figure 2里使用正负号对因素与因素之间的正相关、负相关进行了表示。在朱诺的相关调查中表明,过量的旅游人数一方面会带来本地人收入的增加,另一方面也会给基础设施带和环境带来压力。但同时,旅游人数的增加又会给政府带来相当可观的税收,可以用于维护基础设施和进行环境建设。另一方面,通过THF(tourism heat footprint)的计算方法,我们可以算出游客的平均AHR(Anthropogenic heat release),通过AHR从而进行冰川消融的预测。冰川消融也可能会导致游客满意度的下降,而环境、基础设施的破坏会同时减少本地人和游客的满意度。因此通过使用多目标优化的模型,以旅游人数作为自变量,我们可以得到pareto最优解集,解集范围内的人数即为相对理想、能发展可持续旅游的游客人数。
 \subsection{建立数学模型}
 \begin{enumerate}[(1)]
    \setlength{\parsep}{0ex} %段落间距
\setlength{\topsep}{2ex} %列表到上下文的垂直距离
\setlength{\itemsep}{1ex} %条目间距
\item 税跟游客数量的数学模型\\
旅游税的增加在一定程度上会影响游客的数量。从文献\cite{taxandtour}中我们可以得出旅游税每增加 10\%旅游人数就减少 5.4\%的结论。利用该结论并结合文献\cite{pop}中的数据,我们可以建立数学模型:\\
$$C=1670000 \times 0.946^{10x-48}$$\\
其中C为游客数量,x为税率。从途径上,增加旅游税可以在一定程度上限制游客的人数,在控制游客数量的时候可以考虑增加税款,但是由于模型相对简陋,对于过高或者过低的税率的情形没有作出很好的预测,所以在这里限制旅游税率的范围:$$3\% \leq x  \leq 20\%$$
这里我们可以得到一个可观的曲线图:
\begin{figure}[htbp]  %h此处,t页顶,b页底,p独立一页,浮动体出现的位置 [H]
 
    \centering  %图表居中
    \includegraphics[width=1\textwidth]{model1.png} %图片的名称或者路径之中有空格会出问题 
    \caption{Relationship between Tourist Count and Tax Rate} % 图片标题                 
    \label{fig1}%交互引用
    \end{figure}
\item 游客满意度的数学模型\\
通过前面建立的因果模型,我们定义游客满意度分为CIS和CTS。对于CIS,我们定义:
$$CIS=F(IS)$$
$$IS=CDI+TTI$$
其中的函数F时将State of infrastructures转化为CIS的函数,后面会进行解释。通过查阅政府报告\cite{Inf},我们得到government\textquotesingle s EIE和CDI。
对数据进行拟合,我们得到CDI的数学模型:
$$
CDI=
\begin{cases}
    -12C^{1.05}, &if\ C\ is\ no\ more\ than\ 15000\\
    -12C^{1.06}, &if\ C\ is\ more\ than\ 15000
\end{cases}
$$
CDI的图线如下:
对于税款我们可以通过报告\cite{spend}中得到的E和x求出:
$$T=C\times E\times x$$
根据报告中的数据以及往年对于基础设施上的支出,我们最终拟合出了TTI的函数。
$$TTI=
\begin{cases}
    T-150000, &if\ T\ is\ no\ more\ than\ 546000\\
    \frac{25T}{1.2 \log_{1.4}{100x}}, &if\ T\ is\ more\ than\ 546000
\end{cases}
$$
图线如下:
\begin{figure}[htbp]  %h此处,t页顶,b页底,p独立一页,浮动体出现的位置 [H]
 
    \centering  %图表居中
    \includegraphics[width=0.9\textwidth]{tax2inf.png} %图片的名称或者路径之中有空格会出问题 
    \caption{Graph of TTI} % 图片标题                 
    \label{fig1}%交互引用
    \end{figure}
模型展示了税收变多到了一定程度时,在基建上的投入比例会有所降低。
如此,我们可以获得IS的图线:
\begin{figure}[h]  %h此处,t页顶,b页底,p独立一页,浮动体出现的位置 [H]
 
    \centering  %图表居中
    \includegraphics[width=0.9\textwidth]{IS.png} %图片的名称或者路径之中有空格会出问题 
    \caption{Graph of IS} % 图片标题                 
    \label{fig1}%交互引用
    \end{figure}
\clearpage
F 是将IS转化为satisfaction的函数,对于本地人和游客而言都是相同的。我们定义satisfaction为一种评判的指标,其值域范围为-100到100。
$$
CIS(LIS)=
\begin{cases}
    IS\times 2.408\times 10^{-5}-100, &if\ IS\ is\ no\ more\ than\ 4200000\\
    \log_{1.01}{\frac{x-10}{40000}}-470, &11624010\geq IS \geq 4200000
\end{cases}
$$
\end{enumerate}
 \section{A causal model of the impact of tourist numbers on satisfaction}
 \section{Multi-Objective Optimization Model}
 
 You need to several sections to model, solve, evaluate and predict each task or problem. In the following, we shall review some helpful information to this respect.
 
 \subsection{Analysis of the problem}
 
 \subsection{Modelling}
 
 

 \section{Sensitivity Analysis}
 \subsection{人均基建开销变化}
 
 
 \begin{figure}[H]  %h此处,t页顶,b页底,p独立一页,浮动体出现的位置 [H]
 
 \centering  %图表居中
 \includegraphics[width=.9\textwidth]{sensitivity12.png} %图片的名称或者路径之中有空格会出问题 
 \caption{The optimal tourist count at an infrastructure maintenance expense per tourist of \$8} % 图片标题                 
 \label{figx}%交互引用
 \end{figure}

 
 \begin{figure}[H]  %h此处,t页顶,b页底,p独立一页,浮动体出现的位置 [H]
 
 \centering  %图表居中
 \includegraphics[width=.9\textwidth]{sensitivity8.png} %图片的名称或者路径之中有空格会出问题 
 \caption{The optimal tourist count at an infrastructure maintenance expense per tourist of \$8} % 图片标题                 
 \label{figy}%交互引用
 \end{figure}

 如图所示,在人均基建开销由\$12减少到\$8时,最优人数值并未发生变化,均维持在15000左右,
 Therefore, it can be considered that the optimal 游客人数 is relatively insensitive to 人均基建开销 and has strong robustness. 
 \subsection{增加税率}
 - 我们把船票价格提升20\%,额外的收入用于维护旅游环境;结果显示,可持续旅游人数限制上升至15620人,表明额外的船税对维护可持续旅游有轻微积极影响;
   - 我们提升了3\%的综合税率,结果显示,可持续旅游人数限制上升至16060人,表明额外的综合税率对维护可持续旅游有中度影响;但是,由于维护环境,特别是保护冰川的边际效益较小,更高的税率对提升可持续旅游人数限制不产生显著作用,反而可能减少游客的旅游意愿。
 \subsection{环保成本降低}
 投放保护环境的宣传广告可能有利于维护旅游环境。根据资料,我们假定,政府支出部分款项用于宣传,从而减少维护每个游客所需要的费用。结果表明,仅需4\%的船税提升,可持续旅游人数限制上升至16620人,表明投放保护环境的宣传广告相比直接投入环境保护,在中等规模宣传下有显著积极作用。因此,我们推荐政府投放广告,倡议游客保护环境。
 \begin{itemize}
     \setlength{\parsep}{0ex} %段落间距
     \setlength{\topsep}{2ex} %列表到上下文的垂直距离
     \setlength{\itemsep}{1ex} %条目间距
     \item Sensitivity analysis provides users of mathematical and simulation models with tools to appreciate the dependency of the model output from model input and to investigate how important is each model input in determining its output.
 \end{itemize}
 
 \section{Strengths and weaknesses}
 
 \subsection{Strengths}
 
 \begin{itemize}
     \setlength{\parsep}{0ex} %段落间距
     \setlength{\topsep}{2ex} %列表到上下文的垂直距离
     \setlength{\itemsep}{1ex} %条目间距
     \item 在causal模型中充分考虑了游客人数对居民收入、政府税收、基础设施状态和生态环境的影响
     \item 适当修改后的模型能够被用于另一个受过度旅游影响的目的地,具有通用性
     \item 在求解多目标优化问题时使用了NSGA-II遗传算法,该算法结合了神经网络和符号推理,具有鲁棒性	
 \end{itemize}
 
 \subsection{Weaknesses}

 \begin{itemize}
    \setlength{\parsep}{0ex} %段落间距
     \setlength{\topsep}{2ex} %列表到上下文的垂直距离
     \setlength{\itemsep}{1ex} %条目间距
      \item 没有考虑突发自然灾害对Juneau旅游业的影响
      \item 算法没有将保护冰川之外的旅游资源作为约束条件
      \item 没有考虑政府资金在环境保护和基础设施维护上的分配比例随税收总额的变化
 \end{itemize}
 
 
 \clearpage
 %另起一页继续写。这时,你最好使用“\clearpage” 
 \section{A Memo to the tourist council of Juneau}
 \noindent
 \textbf{Date:} January 27th, 2025

 \noindent
 \textbf{To:} Tourist Council of Juneau

 \noindent
 \textbf{From:} MCM Team \#2511654

 \noindent
 \textbf{Subject: A Multi-Objective Optimization Model for Adjusting the Daily Number of Tourists}
 

 {\noindent}	 \rule[-0pt]{16.5cm}{0.15em}\\
\noindent
\textbf{Dear members of the council:}

针对朱诺市过度旅游的现状,我将向您展示我们建立的优化模型和预测结果。\\[1.5em]
\textbf{·我们的模型和优势}:\\
\textbf{1.考虑因素全面的因果模型:}我们综合分析了游客人数对居民收入、政府税收、基础设施状态和生态环境的影响,根据可靠数据进行数学建模\\
\textbf{2.结果可靠的多目标优化模型:}我们建立可靠的游客满意度和居民满意度函数,运用遗传算法求得游客数量的Pareto解,最大化利益相关方的满意度
\\[1.5em]
\textbf{·我们得出的预测结果:}
经过求解多目标优化模型,我们推荐在当前状态下,控制旅游人数在14550人左右,以推进可持续旅游
\\[1.5em]
\textbf{·我们的建议措施:}\\
\textbf{1.适度提升税率:}提升范围在3\%至20\%有利于提高维持旅游业可持续发展的游客限额\\
\textbf{2.环保宣传:}支出部分款项用于投放保护环境的宣传广告,倡议游客保护环境,能降低投入环境保护的经济成本
 % 参考文献,务必统一格式,下面以书籍、期刊文章、网页资料为例
 \clearpage   %另起一页继续写。这时,你最好使用“\clearpage” 
 
 \begin{thebibliography}{99}
    \bibitem{taxandtour}
    Adedoyin, F. F., Seetaram, N., Disegna, M., \& Filis, G. (2023). The Effect of Tourism Taxation on International Arrivals to a Small Tourism-Dependent Economy. Journal of Travel Research, 62(1), 135-153. 
     
    \bibitem{pop}
    Formery McDowell Croup, ECONOMIC IMPACT OF JUNEAU\textquotesingle S CRUISE INDUSTRY 2023
    {https://juneau.org/wp-content/uploads/2024/01/CBJ-Cruise-Impacts-2023-Report-1.22.24.pdf}

    \bibitem{spend}
    Juneau Tourism Survey 2023 \& Juneau Cruise Passenger Survey 2023 \& Juneau Cruise Industry Impacts 2023 Extended Slide Deck, McKINLEY RESEARCH
    {https://www.seconference.org/wp-content/uploads/2024/02/Tourism-Panel-Heather-Haugland.pdf}

    \bibitem{Inf}
    Biennial Budget of Juneau \href{https://juneau.org/wp-content/uploads/2024/08/FY25-Adopted-Budget-Book-Final.pdf}{https://juneau.org/wp-content/uploads/2024/08/FY25-Adopted-Budget-Book-Final.pdf}

   %% 书籍
   \bibitem{NDZY2021}
   H. Ni, X. Dong, J. Dong, G. Yu, An Introduction to Machine Learning in Quantitative Finance, World Scientific, 2021.
   
   %% 期刊文献
   \bibitem{V2020}
   P. Virtanen et al., SciPy 1.0: fundamental algorithms for scientific computing in
   Python, Nature Methods, 17, 261--272, (2020).
   
   %% 网页资料
   \bibitem{Mesevage2021}
   T.G. Mesevage, What is data preprocessing \& What are the steps involved? (2021) \href{https://monkeylearn.com/blog/data-preprocessing/}{https://monkeylearn.com/blog/data-preprocessing/}
 \end{thebibliography}
 
 % \includepdf[pages={1,2}]{Memo.pdf} 
 
 \end{document}  % 结束